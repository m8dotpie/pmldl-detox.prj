\section{Datamining}

Primary sources of the information were:

\begin{enumerate}
    \item ParaNMT filtered corpus
    \item Glove embeddings
    \item WordNet
    \item Jigsaw bad words dataset
\end{enumerate}

\subsection{ParaNMT}

While the dataset was considered as a dataset of the toxic sentences and their
non-toxic paraphrases with toxicity ranking from people, it seems to be
slightly wrong.

It seems that paraphasing is the reverse-translation of the sentence, which was
done artificially as well.

Moreover, original work for toxicity ranking proposed one more model.

As a result, the main dataset was artificially generated, artificially ranked
and in general is not a very good source of data.

\subsubsection{Filtering the filtered}

To achieve decent results we can carefully pick the samples from the provided
dataset. For instance, experimentally we deduced that filtering the samples
which lead from high level of toxicity to the very low in result, filters many
insane examples.

Moreover, such filter is beneficial in terms of actually successfully
detoxifying the text, while preserving the high similarity of the reference and
translation.

Final parameters for filtering:
\begin{equation}\text{tox}_\text{ref} > 0.99 \wedge \text{tox}_\text{trn} < 0.01\end{equation}

\subsubsection{Splitting}

After filtering the data, we can deterministically sample the data with
specified random seed, to achieve reproducible results. For instance, we want
\(3000\) samples for test data. \(10000\) of samples will be split in \(9:1\)
correspondence for train/validation datasets. While larger samples are
computationally feasible, they tend not to improve final results greatly, as a
result, we will stick with these values.

